\documentclass{article}
% Damit die Verwendung der deutschen Sprache nicht ganz so umst\"andlich wird,
% sollte man die folgenden Pakete einbinden: 
\usepackage[latin1]{inputenc}% erm\"oglich die direkte Eingabe der Umlaute 
\usepackage[T1]{fontenc} % das Trennen der Umlaute
\usepackage{ngerman} % hiermit werden deutsche Bezeichnungen genutzt und 
                     % die W\"orter werden anhand der neue Rechtschreibung 
		     % automatisch getrennt.
\usepackage{authblk}  

\title{Reinforcement Learning Exercise 1}
\author{Alexander J\"aggle}
\author{Johannes Haberstock}
\author{Daniel Arnold}

\affil{M.Sc. Autonomous Systems, University of Stuttgart}

\date{\today}
% Hinweis: \title{um was auch immer es geht}, \author{wer es auch immer 
% geschrieben hat} und  \date{wann auch immer das war} k\"onnen vor 
% oder nach dem  Kommando \begin{document} stehen 
% Aber der \maketitle Befehl mu\ss{} nach dem \begin{document} Kommando stehen! 
\begin{document}

\maketitle

\section{Multi-armed Bandits}
Ziel der Arbeit ist es m\"oglichsten vielen oder wenn m\"oglichen es allen 
zu erm\"oglichen, Dokumente mit \LaTeX{} zu erstellen!

\begin{itemize}
    \item[a)] The probability of the greedy action being selected is $p = .5$ since $p = 1- \epsilon$.   
    \item[b)] {\begin{itemize}
        \item[1.] random = [{1,2,5}]
        \item[2.] greedy = [{3,4}]
    \end{itemize}}
\end{itemize}


\paragraph{Explanation to b)}
Before step three is executed, only the rewards for action $1$ and action $2$ are known, which both are 1. 
Every other action, which was not explored yet, is assumed with a reward of 0. Thus, at timestep $t = 3$, 
action $2$ with a reward of 1 was selected at greedy. 

A similar scenario happened at timestep $t = 4$, when action-value estimates $Q$ are known. For timestep $t = 3$,
the action-value estimate is 1.5, while the other two estimates are 1 respectively 0. Since the next action, which
was selected was action $2$, and thus it was greedy. 


\section{Dokumentenklassen} \label{documentclasses}

\begin{itemize}
\item article
\item book 
\item report 
\item letter 
\end{itemize}


\begin{enumerate}
\item article
\item book 
\item report 
\item letter 
\end{enumerate}

\begin{description}
\item[article\label{article}]{Article ist \ldots}
\item[book\label{book}]{Die book Klasse ist \ldots}
\item[report\label{report}]{Die Klasse report erm\"oglicht es  \ldots}
\item[letter\label{letter}]{Wenn man einen Breif schreiben sollte man eine 
	andere Klasse nutzen, da diese f\"ur ein anderes als das deutsche 
	Briefformat ausgelegt ist.}
\end{description}


\section{Fazit}\label{conclusions}
Nach langer Suche hat sich herausgestellt, dass es kein l\"angeres 
\LaTeX{} Beispiel, als das von \cite{doe} geschriebene gibt. 

\end{document}